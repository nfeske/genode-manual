%%
%% Tikz styles used for diagrams
%%

\tikzstyle{every node}+=[font=\footnotesize]

\tikzstyle{dropshadow} = [blur shadow={shadow blur steps=5,shadow xshift=.0ex,
                                       shadow yshift=-0.3ex,opacity=0.9,
                                       shadow blur radius=0.5ex}]

\tikzstyle{compound} = [rectangle, draw, text centered,
                         rounded corners,
                         top color=white,
                         bottom color=black!5,
                         dropshadow,
                         draw=black!10]

\tikzstyle{component} = [compound, draw=black!70]

\tikzstyle{compoundlabel} = [below right,
                             text opacity=0.6, inner sep=0, outer sep=0.7ex]

\tikzstyle{capability} = [circle, draw, fill, thick,
                          inner color=white,
                          outer color=blue!30,
                          inner sep=1pt,
                          minimum size=2.8ex,
                          fill opacity=1.0, dropshadow,
                          anchor=mid]

\tikzstyle{capslot} = [minimum size=2.5ex, inner sep=0ex, outer sep=0ex,
                       fill=white, draw=black!70]

\tikzstyle{kernelobj} = [draw, tape, tape bend top=none,
                         draw=black!70, align=center, fill=white, dropshadow]

\tikzstyle{arrow} = [draw=black!80, ->, >= stealth, sloped, above]

\tikzstyle{userland} = [fill=green]

\tikzstyle{kernel} = [fill=yellow]

\tikzstyle{thread} = [font=\normalsize]

\tikzstyle{nodecompound} = [rectangle, draw, align=center,
                            minimum height=2em, minimum width=2em,
                            inner sep=1.5ex, outer sep=1ex,
                            rounded corners,
                            top color=white,
                            bottom color=black!20,
                            dropshadow,
                            draw=black!10, thick]

\tikzstyle{ownership} = [draw, single arrow, draw=black!10, minimum height=4ex,
                         bottom color=white, top color=red!70,
                         single arrow head extend=1ex]

\tikzstyle{every picture}+=[remember picture]

\newcommand\kernelred{black!20!red!100}

\tikzstyle{dataspace} = [tape, draw, tape bend top=none, fill=white,
                         dropshadow, minimum height=6ex, align=center,
                         tape bend height=0.5ex, fill opacity=0.95]

% dotted line that connects dataspaces mapped in different address spaces
\tikzstyle{dataspacemapped} = [draw, draw, dotted]

% style representing a hardware component
\tikzstyle{hardware} = [draw=black!50, dropshadow, align=center,
                        path fading=flow fade, thin,
                        top color=black!5, bottom color=black!20]

%%
%% Styles for flow charts
%%
\tikzstyle{flowborder} = [draw=black!50]

\tikzstyle{flowbox} = [flowborder, fill=white, dropshadow,
                       rounded corners=0.5ex,
                       fill=yellow!20,
                       path fading=flow fade]

\tikzstyle{flowdecision} = [flowborder, diamond, fill=blue!20,
                           text badly centered, align=center,
                           inner sep=0.5ex,
                           dropshadow, path fading=flow fade]

%%
%% Styles for timeline diagrams
%%

\tikzstyle{activetimeline} = [draw,
                              blur shadow={shadow blur steps=5},
                              shadow yshift=-0.3ex, shadow xshift=0em,
                              rounded corners=0.4ex,
                              color=black!30,
                              path fading=timeline fade right,
                              left color=yellow!90!white!50,
                              right color=yellow!100!white!50,
                              minimum width=2.5ex]

\tikzstyle{timelinemessage} = [draw=black!80, ->, >= stealth, sloped,
                               shorten >= 0.8ex, shorten <= 0.8ex]

\tikzstyle{timelinegroup} = [draw=black!50, dashed,
                             rounded corners, fill=black, fill opacity=0.05]


%%
%% Styles for UML class diagrams
%%

\tikzstyle{umlclass}  = [draw, fill=white, dropshadow, align=left]
\tikzstyle{umlclass2} = [umlclass, rectangle split, rectangle split parts=2]
\tikzstyle{umlclass3} = [umlclass, rectangle split, rectangle split parts=3]

\tikzstyle{umlinfo} = [umlclass,
                       chamfered rectangle,
                       chamfered rectangle corners=north east,
                       text opacity=0.5, draw opacity=0.5, align=center]

\tikzstyle{umlinfoline} = [draw, draw opacity=0.5, dotted, thick]
\tikzstyle{umlinherit} = [draw, ->, >= open triangle 60]
\tikzstyle{umlrelated} = [draw, above]
\tikzstyle{umloperate} = [arrow, dashed, above]


%%
%% Functions for constructing Tikz nodes
%%

%%
% Create capability space node
%
% argument 1: node name, also used as prefix for the individual slots
% argument 2: maximum index
% argument 3: node arguments
%
% Uses tikz style "capslot"
%
\newcommand\capspacenode[3]{
	\node[#3] (#1) {
		\begin{tikzpicture}
			\begin{scope}[start chain,node distance=0,rounded corners=0]
				\foreach \i in {0,1,...,#2} {
					\node [on chain,capslot,draw] (#1\i) {};
					\path (#1\i) node[yshift=0.5ex, xshift=-0.5ex, inner sep=0.5ex,
					                  text=black!60] {\tiny \i};
				}
				\node [on chain,capslot, fill=none, draw=none, outer sep=0.5ex] {$\ldots$};
			\end{scope}
		\end{tikzpicture}
	};
}


%%
% Create RPC object node
%
% argument 1: node name, also used as prefix for the individual slots
% argument 2: node arguments
% argument 3: capability name
% argument 4: node content
% argument 5: capability node arguments
%
% The capability node will be named #1cap.
% The object node will be named #1obj.
%
\newcommand\rpcobjectnodebase[5] {
	\node[#2] (#1) {
		\begin{tikzpicture}
			\node [draw, left color=blue!20, right color=white, minimum size=0pt,
			       thick, inner sep=1ex, rounded corners=0.5ex, dropshadow]
			       (#1obj) {#4};
			\node [outer sep=0pt, capability, #5]
			       (#1cap) {#3};
			\path [draw, dropshadow, very thick] (#1cap) -- node (#1link) {} (#1obj);
		\end{tikzpicture}
	};
}


%%
% Create RPC object node with the capability to the right of the object
%
% The four arguments are the same as for 'rpcobjectnodebase'.
%
\newcommand\rpcobjectrnode[4] {
	\rpcobjectnodebase{#1}{#2}{#3}{#4}{right=1.5ex of #1obj} }


%%
% Create RPC object node with the capability to the left of the object
%
% The four arguments are the same as for 'rpcobjectnodebase'.
%
\newcommand\rpcobjectlnode[4] {
	\rpcobjectnodebase{#1}{#2}{#3}{#4}{left=1.5ex of #1obj} }


%%
% Create compounding node
%
% argument 1:  node name
% argument 2:  node style
% argument 3:  node content
%
\newcommand\compoundnode[3] {
	\node[#2] (#1) {
		\begin{tikzpicture}[xshift=0, yshift=0, outer sep=0, inner sep=0]
			#3
		\end{tikzpicture}
	};
}


%%
% Create labeled compounding box node
%
% argument 1:  node name
% argument 2:  node style
% argument 3:  label to appear at the north-west corner
% argument 4:  content of the box
%
\newcommand\labeledcompoundnode[4] {
	\compoundnode{#1}{inner sep=3ex, #2}{#4}
	\path (#1.north west) node[compoundlabel] {#3};
}


%%
% Create kernel-user boundary
%
% argument 1:  nodes that are contained in the kernel,
%              specified as tikz style, e.g., 'fit=(node1) (node2)...'
%
\newcommand\kernelboundary[1] {
	\node [inner sep=4ex, #1] (kernelboundary) {};
	\draw [dashed, very thick, color=\kernelred] (kernelboundary.north west)
	      -- (kernelboundary.north east)
	      node[below left] {kernel};
}


%%
% Create node for an upward pointing thread
%
% argument 1:  node name
% argument 2:  node style
%
\newcommand\upwardthreadnode[2] {
	\node [thread, #2] (#1) {$\uplsquigarrow$}; }


%%
% Create node for an downward pointing thread
%
% argument 1:  node name
% argument 2:  node style
%
\newcommand\downwardthreadnode[2] {
	\node [thread, #2] (#1) {$\downrsquigarrow$}; }


%%
% Create entrypoint node
%
% argument 1:  node name
% argument 2:  node style
% argument 3:  component where the entrypoint resides
%
% The command creates sub nodes for the semicircle called #1ep and the
% thread called #1thread.
%
\newcommand\entrypointnode[3]{
	\node [above, inner sep=0, outer sep=0, #2] at (#3.south) (#1) {
		\begin{tikzpicture}

			\node [draw=black, draw opacity=0.4, ball color=blue, fill opacity=0.2,
			       rounded corners=0, shape=semicircle,
			       inner sep=1.3ex, outer sep=0, above]
			(#1ep) {};

			\path (#1ep.arc end) node[compoundlabel, right, yshift=1ex] {EP};

			\path (#1ep.arc start) [thread]
				node [left, xshift=-0.3ex, yshift=1ex] (#1thread) {$\uplsquigarrow$};

		\end{tikzpicture}
	};
}


%%
% Create path that illustrates a capability delegation
%
\newcommand\delegationpath{
	\path [color=black!30!blue!30, text=black,
	       decoration={markings,
	                   mark=between positions 0.03 and 1 step 1ex with {\arrow{latex}}},
	       postaction={decorate}]}


\newcommand\capability[1]{
	\begin{tikzpicture}[baseline=-0.6ex]
	\node [capability] {#1};
	\end{tikzpicture}
}


%%
%% Timeline diagrams
%%

\newcounter{numtimelines}


%%
% Create new vertical timeline
%
% argument: timeline name
%
\newcommand{\newtimeline}[1] {

	\stepcounter{numtimelines}

	\path (timelineanchor) node (#1)     {};
	\path (timelineanchor) node (last#1) {};

	\pgfkeyssetvalue{/timeline\thenumtimelines/name}{#1}

	\pgfkeyssetvalue{/timeline #1/state}{idle}
}


%%
% Draw vertical state representation of the specified timeline (internal)
%
% argument: timeline name
%
\newcommand{\finishtimeline}[1] {

	\def\state{\pgfkeysvalueof{/timeline #1/state}}

	\ifthenelse{\equal{\state}{idle}}{

		\draw[dashed, thick, draw opacity=0.1] (last#1) -- (#1);
	}{}

	\ifthenelse{\equal{\state}{active}}{

		\path (last#1)+(0,-0.1) node (top)    {};
		\path     (#1)+(0, 0.1) node (bottom) {};

		\path (last#1) node[inner sep=0pt, fit=(top) (bottom), activetimeline] {};
	}{}

	\path (#1) node (last#1) {};
}


%%
% Timeline environment
%
% The environment can only be used within a tikzpicture environment
%
\newenvironment{timelinediagram} {

	% reset timeline counter
	\setcounter{numtimelines}{0}

	% reset anchor used for next call of 'newtimeline'
	\path (0,0) node (timelineanchor) {};

	\newcommand{\timestepvector}{0,-1}

	%%
	% Add time step
	%
	\newcommand{\timestep} {

		\foreach \i in {1,...,\thenumtimelines}{

			\def\name{\pgfkeysvalueof{/timeline\i/name}}

			% move timeline forward in time
			\path (\name)+(\timestepvector) node (\name) {};
		}
	}

	%%
	% Activate timeline
	%
	\newcommand{\activate}[1] {
		\finishtimeline{##1}
		\pgfkeyssetvalue{/timeline ##1/state}{active}
	}

	%%
	% Deactivate timeline
	%
	\newcommand{\deactivate}[1] {
		\finishtimeline{##1}
		\pgfkeyssetvalue{/timeline ##1/state}{idle}
	}

	%%
	% Create transition from one timeline to another
	%
	\newcommand{\transition}[3] {
		%
		% The named 'transitionlabel' node can be used to attach further
		% information to the message. The unnamed second node is designated
		% for labeling simple transitions.
		%
		\draw[timelinemessage] (##1) -- node (transitionlabel) {} node[above] {##3} (##2);
		\deactivate{##1}
		\activate{##2}
	}

} {
	% finish all timelines
	\foreach \i in {1,...,\thenumtimelines}{
		\def\name{\pgfkeysvalueof{/timeline\i/name}}
		\finishtimeline{\name}
	}
}
